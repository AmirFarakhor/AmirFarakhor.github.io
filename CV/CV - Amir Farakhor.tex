%%%%%%%%%%%%%%%%%%%%%%%%%%%%%%%%%%%%%%%%%%%%%%%%%%%%%%%%%%%%%%%%%%%%%%%%%%%%%%
%                             Curriculum Vitae                               %
%%%%%%%%%%%%%%%%%%%%%%%%%%%%%%%%%%%%%%%%%%%%%%%%%%%%%%%%%%%%%%%%%%%%%%%%%%%%%%
\documentclass[10pt,letter]{article}
\topmargin-2.0cm
\advance\oddsidemargin-1in
%\advance\evensidemargin-1.2cm
\textheight9.2in
\textwidth6.75in
\newcommand\bb[1]{\mbox{\em #1}}
%\def\baselinestretch{1.25}
\def\baselinestretch{1}

\usepackage{multicol}
% The use of the times package forces the use of the type-1 times
% roman font, but the times roman font does not look nice.
% Besides the times roman font still does not print correctly on
% the dopy printer.
%\usepackage{times}

\usepackage{fancyhdr}
\usepackage{origpagecounting}
\usepackage[dvips]{color}
\usepackage{hyperref}

\newcounter{myEnumCounter}
\newcounter{mySaveCounter}
\renewenvironment{enumerate}{%
  \begin{list}{\arabic{myEnumCounter}.}{\usecounter{myEnumCounter}%
  \setcounter{myEnumCounter}{\value{mySaveCounter}}}
  }{%
  \setcounter{mySaveCounter}{\value{myEnumCounter}}\end{list}%
}
\newcommand\myEnumReset{\setcounter{mySaveCounter}{0}}

% The old enumerate environment is rewritten, so you need no special command to
% start continuing counting. With the command \myEnumReset you can Reset the couter
% at any place in the text.

% http://www.educat.hu-berlin.de/~voss/lyx/list/enum.phtml

\definecolor{gray}{rgb}{0.4,0.4,0.4}

\begin{document}

%\thispagestyle{empty}
%\pagestyle{plain}

%\thispagestyle{fancy}
%\pagenumbering{gobble}
%\fancyhead[location]{text}
% Leave Left and Right Header empty.
%\lhead{\textcolor{gray}{\it Sundar Iyer}}
%\rhead{\textcolor{gray}{\thepage/\totalpages{}}}
%\rhead{\thepage}
\renewcommand{\headrulewidth}{0pt}
\renewcommand{\footrulewidth}{0pt}
\fancyfoot[C]{\footnotesize \textcolor{gray}{}}
%A copy of this curriculum vitae, publications and 
%talk slides are available for download at
%http://www.stanford.edu/$\sim$sundaes/application}}


%\pagestyle{myheadings}
%\markboth{Sundar Iyer}{Sundar Iyer}

\vspace*{-0.75cm}
\begin{center}
{\huge \bf AMIR FARAKHOR}
\vspace*{0.25cm}
\end{center}

\begin{small}

%===================================
\begin{tabbing}
\=xxxxxxxx\=xxxxxxxx\=xxxxxxxx\=\kill
\begin{tabular*}{\linewidth}{l@{\extracolsep{\fill}}r}

2900 Bob Billings Parkway & Phone: (785) 979-3425 \\
Lawrence, KS-66049 &  Email: \href{mailto:a.farakhor@ku.edu}{a.farakhor@ku.edu}\\\href{https://amirfarakhor.github.io/}{https://amirfarakhor.github.io/}
%http://www.stanford.edu/$\sim$sundaes & Alt: sundaes@cs.stanford.edu \\
\end{tabular*}
\end{tabbing}

\vspace*{0.2cm}


%==========================================
%\vspace{0.20cm}

\subsection*{PARTICULARS}

%{\color{DarkSeaGreen}} \hrule
\hrule
\vspace{0.2cm}
%%%%%%%%%%%%%%%%%%%%%%%%%%%%%%%

\subsubsection*{EDUCATION}
%\vspace{0.2cm}

%\begin{tabbing}
%xxxxxxxx\=xxxxxxxx\=xxxxxxxx\=xxxxxxxx\=\kill
%\>{\bf Academic record}\\[.7em]

%\>\begin{tabular}{|l|l|l|}
%\hline
%Certificate	&	Place of study	&	Year\\
%\hline
%Ph. D. in Computer Sc.  & Stanford University     & {\it 2000-, Defended 2003} \\
%\hline
%M. S. in Computer Sc. & Stanford University     & {\it June 2000} \\
%\hline
%B. Tech in Computer Sc. and Engg. & I.I.T. Bombay  &	{\it April 1998} \\
%\hline
%\end{tabular}
%\end{tabbing}

%xxxx\=xxxxxxxx\=xxxxxxxx\=xxxxxxxx\=\kill
\renewcommand{\arraystretch}{1.2}

\begin{tabbing}
xxxx\=xxxxxxxx\=xxxxxxxx\=xxxxxxxx\=\kill
%\>\begin{tabular*}{6.1in}{lr}

\>\begin{tabular*}{0.9\linewidth}{l@{\extracolsep{\fill}}r}
University of Kansas & Lawrence, KS \\
Ph. D. in Mechanical Engineering  &  {\it Jan 2021 - Present}\\
\textbf{GPA}: 4.00 & \\
\parbox[l]{12cm}{\vspace{3pt}\textbf{Thesis Topic}: On Advanced Battery Energy Storage Systems: Design, Optimal Control, and Experimentation\vspace{3pt}} & \\ 
\textbf{Supervisor}: Dr. Huazhen Fang & \\
 & \\

University of Tabriz & Tabriz, Iran \\
Ph. D. in Electrical Engineering - Power Electronics & {\it Sep 2015 - Feb 2019}\\
\parbox[l]{11.5cm}{\vspace{3pt} \textbf{Thesis Topic}: Design and Derivation of New Power Electronic Converters For Renewable Energy Sources \vspace{3pt}} & \\ 
\textbf{Member of Organization Exceptional Talents} of University of Tabriz & \\
 & \\

Azarbaijan Shahid Madani University & Tabriz, Iran \\
M. Sc. in Electrical Engineering & {\it  Sep 2012 - May 2014} \\
\textbf{Member of Organization Exceptional Talents} of Azarbaijan Shahid Madani University & \\
\\
Azarbaijan Shahid Madani University & Tabriz, Iran \\
B. Sc. in Electrical Engineering & {\it Sep 2008 - Sep 2012} \\
\textbf{Member of Organization Exceptional Talents} of Azarbaijan Shahid Madani University & \\

\end{tabular*}
\end{tabbing}

\subsubsection*{CURRENT STATUS}
\begin{list}{}{}
\item \textbf{U.S. Permanent Resident}, Citizen of Iran.
\end{list}

%\subsubsection*{RESEARCH INTERESTS}
%%\hrule
%%\vspace{0.2cm}
%
%\begin{list}{}{}
%\item My research interests span the areas of network algorithms, system architecture
%and component design. I have a specific interest in algorithms for packet processing,
%switching, router design, memory architectures and scalable load balancing systems.
%\end{list}

%\subsubsection*{DISSERTATION}
%
%\begin{list}{}{}
%\item Title: ``Parallelism and load-balancing for the Internet"  \\
%Advisor: Prof. Nick McKeown
%\end{list}
%
%My thesis develops an analytical framework for the design of networking systems
%(specifically memory intensive aspects of Router Architecture, Packet Processing and Switching Algorithms) which give performance guarantees via the use of load balancing algorithms and parallelism in design. 

%\vspace{0.1cm}
%\subsection*{ACADEMIC HONORS}
%\hrule
%\vspace{0.2cm}
%\begin{itemize}
%%==================
%
%	\item Cisco Stanford FMA Fellowship, Stanford University, 2002-2003.
%
%	\item Siebel Scholars Fellowship, Stanford University, 2001.
%
%        \item  Paper chosen amongst the best papers from IEEE Hot Interconnects 2001 and
%selected for publication to IEEE Micro.
%
%	\item Christofer Stephenson Memorial Award for the 
%    best Masters Thesis in Computer Science, Stanford University, 2000.
%
%	\item Indian National Talent Search Merit Scholarship, 1992-1998.
%
%	\item Indian National Talent Search Examination in Physics (NSEP) awardee, 1993.
%
%	\item  Selected to participate in the Indian National Maths Olympiad, 1993.
%
%    \item Indian National Merit Scholarship, 1992-1998.
%
%\end{itemize}

%\newpage
\pagestyle{fancy}
%\lhead{\textcolor{gray}{\it Sundar Iyer}}
%\rhead{\textcolor{gray}{\thepage/\totalpages{}}}
\fancyfoot[C]{}

%\vspace{0.1cm}
%\subsection*{WORK EXPERIENCE}
%\hrule
%\vspace{0.2cm}
%\begin{itemize}
%\item {\bf CTO, Co-Founder and Member of the Board, Nemo Systems,} Nov
%2003 - Sep 2005. Nemo Systems (acquired by Cisco Systems in Sep 2005) 
%was a fabless semiconductor company, building caching memory sub-systems 
%for networking applications. 
%
%\item {\bf Consultant, Nevis Networks Inc.}, Sep. 2003 - Mar 2004.
%Analyzed the performance of the Nevis switch fabric. Suggested
%theoretically optimal and practical packet switching algorithms for the
%distributed shared memory switch fabric.
%
%\item {\bf Systems Engineer, PMC-Sierra,} Sep. 2000 - Oct. 2001.
%
%\item {\bf Senior Systems Architect, SwitchOn Networks,} Jan. 1999 - Sep. 2000.
%Part of the initial team of six members of SwitchOn Networks (later acquired by PMC-Sierra in Sep. 2000) during
%its founding. Jointly responsible for the design, architecture, and algorithm evaluation of the ClassiPI chip set.
%ClassiPI was a networking content co-processor which was released commercially in Feb. 2001. Other joint
%responsibilities included technical marketing and patent strategy.
%
%\item {\bf Consultant, RIMO Technologies, India}, Jun. 1998 - Dec. 1998. \\
%
%%  Other joint responsibilities included technical marketing, patent strategy,
%%  business development and sales. Was one of the first six founding members, of 
%%  SwitchOn Networks (now acquired by PMC-Sierra) and was involved in this startup 
%%  from the initial concept and venture capital stage to the successful acquisition and
%%  commercial release of the ClassiPI chipset in Feb. 2001.
%
%\end{itemize}

%\vspace{0.1cm}
%\subsection*{RESEARCH EXPERIENCE}
%\hrule
%\vspace{0.2cm}
%\begin{itemize}
%\item {\bf Research Assistant, Stanford University}, Sep 1998 - Present. \\
%{\bf Parallel Packet Switches}, (Sep 1998 - June 2000): We ask the question as to whether it is possible to build a large switch
%from multiple smaller switches stacked in parallel, called the parallel packet switch (PPS), 
%such that it behaves exactly like a large switch. I designed an algorithm and derived the
%first analytical proof of the conditions under which this is possible. I then extended 
%this work so that the PPS can support quality of service and multicast traffic. Later, I derived
%and analyzed a practical distributed algorithm for the PPS. This research demonstrates how switching capacity can be scaled in an efficient
%manner.
%
%\vspace{0.1cm}
%{\bf Distributed Shared Memory Routers}, (Jan 2001 - Mar 2002): From a network designer's
%perspective a shared memory router is ideal in that the packets
%are stored in a central location and the memory bandwidth and space is shared across
%packets from all ports. This sharing helps in conserving memory space and results in low cost
%and low power routers. However, it is a widely held myth that such shared memory routers are not
%scalable to higher speeds due to limitations imposed on the speed of a single memory. I worked 
%on the first demonstration and proof to show how to build and scale the capacity of 
%shared memory routers using distributed memories.  
%
%\vspace{0.1cm}
%{\bf The Pigeon Hole Principle for Routers}, (Sep 1998 - Mar 2002): In the course of our
%work on routers we derived a technique called ``Constraint Sets". This was used to
%analyze a number of router architectures especially those with a single stage of
%buffering. The ``Constraint Set" technique is a generalization of the
%Pigeon Hole principle. Later, I extended this work to Combined Input Output Queued
%routers, which are routers with two stages of buffering.
%
%\vspace{0.1cm}
%{\bf Scalable, High Performance Packet Buffers}, (Sep 2000 - Mar 2002): Packet buffers on
%Internet Routers have two requirements. They need to be large and have to be accessed at a very high rate. Unfortunately commodity memories such as
%DRAMs have slow access speeds though they allow large storage, while commodity SRAMs allow fast
%access but are inefficient in storage. I analyzed a well-known hierarchical buffer architecture
%that has a small amount of SRAM cache combined with DRAM. However, unlike computer architecture
%where caching only allows statistical guarantees, the algorithms proposed exploit the
%characteristics of memory requirements for networking
%to enable the design of a memory architecture which gives deterministic latency
%guarantees. I showed how the optimal algorithm can be modeled using difference equations and used adversarial 
%traffic patterns to derive bounds on the minimum size of the SRAM memory.
%This resulting memory architecture supports
%the access speeds of SRAM and has the storage capacity of DRAM. 
%
%
%\vspace{0.1cm}
%{\bf Deterministic Architectures for Statistics and State Maintenance}, (Sep 2000 - Mar 2002): A number
%of packet processing tasks in the Internet put a large demand on the costs and the power budget for designers. We specifically evaluate two such
%tasks; namely maintaining statistics counters and maintaining state of connections. We
%propose architectural solutions and derive optimal algorithms. Our analytical results are based on the 
%use of potential functions to model the system. This research shows how the above packet processing 
%tasks can be scaled to very high line rates.
%
%\vspace{0.1cm}
%{\bf Distributed Algorithms for Buffered Crossbars}, (June 2002 - Present):  Internet core routers have been designed lately using a crossbar switching fabric. 
%While it is possible to build routers using crossbars which give deterministic performance
%guarantees, the algorithms required to realize this are centralized and have high
%computational complexity. Hence in practice most routers do not
%use these algorithms and do not give any guaranteed performance.
%A colleague of mine, Shang-Tse Chuang,  and I analyzed a slight modification to the
%crossbar fabric called the buffered crossbar, which is simply a crossbar with a small number of buffers in it. 
%We derive a bijection between a crossbar and a buffered crossbar to show that any
%algorithm on the buffered crossbar can be implemented on the crossbar and vice versa. We
%derive a suite of distributed algorithms on the buffered crossbar and derive
%analytically the conditions under which they can give both statistical and deterministic
%guarantees for both first come first serve routers and routers which support quality of
%service. Since these distributed algorithms can operate independently on each input and
%output port of the switch without communicating with each other, they are readily implementable.
%Our results show that Internet routers built using crossbars can be re-designed in a
%practical manner using buffered crossbars and give superior performance and
%deterministic guarantees.
%
%\vspace{0.1cm}
%{\bf Stability Properties of the Maximum Size Matching}, (May 2002 - Present):
%Contrary to intuition, it is known in queueing theory that a greedy policy that
%maximizes the instantaneous throughput in a system of queues may not maximize the long term throughput. 
%However, greedy policies are of interest because they are usually easy to implement.
%A maximum size matching is an example of such a greedy scheduling policy on crossbar
%switches. I analyzed this greedy policy and showed that with 
%stochastically constrained traffic and the framework of scheduling packets in batches,
%the greedy policy maximizes the long term throughput. A number of open
%problems remain in this area and at the time of writing, this work is in progress.
%
%\item {\bf Research Intern, ATL SprintLabs}, June - Sep. 2001. \\
%{\bf Internet Measurements \& Deflection Routing:} Our work involved designing a practical solution for tackling link overload on the network
%backbone. First, I analyzed how overload occurs on the Sprint backbone. 
%This involved writing a tool to characterize link overload and create
%statistics from SNMP measurements collected for a period of a year from the
%Sprint backbone. With the intuition gained from these statistics we concluded that there
%was a need to alleviate overload using deflection routing.
%I suggested a method of setting link weights on the Sprint network (taking advantage of
%the topology characteristics) to enable a loop free deflection routing algorithm.
%
%\item {\bf Research Intern, RIMO Technologies, India}, May - July 1997. \\
%{\bf Packet Classification Algorithms:} Designed classification algorithms and developed a commercial software packet classifier for Windows NT and the VXL Terminal Server embedded system platform.
%
%\item {\bf Undergraduate Research Intern, I. I. T. Bombay}, Apr - July 1996. \\
%{\bf Java Application Software:} Developed algorithms and software for Java applications for distributed computing, under Prof. Sharat Chandran.
%
%\end{itemize}

%\begin{itemize}
%
%\item {\bf Teaching Assistant.} EE384X: Packet Switching Architectures-I, Prof. Balaji
%Prabhakar and Prof. Nick McKeown, Winter 2002, Stanford University.
%
%\item {\bf Teaching Assistant.} EE384Y: Packet Switching Architectures-II, Prof. Balaji
%Prabhakar and Prof. Nick McKeown,
%Spring 2001 and Spring 2002, Stanford University.
%
%\item {\bf Instructor.} Networking, Center for Development of Advanced Computing (CDAC) at MET Mumbai, Winter 1997,
%Bombay, India (jointly with Prof. A. Karandikar, I. I. T. Bombay).
%
%\end{itemize}


% List of papers and publications and patents ?
%===========================================
%\vspace{0.1cm}
\subsection*{PUBLICATIONS}
\hrule
\vspace{0.2cm}

\subsubsection*{Google Scholar Profile}
\begin{itemize}
	\item Total Citations: 1245, H-index: 14, Link: \href{https://scholar.google.com/citations?user=lHKTSM8AAAAJ&hl=en}{Amir Farakhor}
	%\item H-index: 14
	%\item Link: \href{https://scholar.google.com/citations?user=lHKTSM8AAAAJ&hl=en}{Amir Farakhor}
	%\url{https://scholar.google.com/citations?user=lHKTSM8AAAAJ&hl=en}
\end{itemize}

\subsubsection*{JOURNAL PAPERS: To be submitted}

\begin{enumerate}
    \item 
Efficient Optimal Power Management for Battery Energy Storage Systems via Bayesian Inference

\textbf{Amir Farakhor}, Di Wu, Yebin Wang, Huazhen Fang

{\it IEEE Transactions on Control Systems Technology}

    \item 
Economic Optimal Power Management of Second-Life battery Energy Storage Systems

\textbf{Amir Farakhor}, Di Wu, Huazhen Fang

{\it IEEE Transactions on Sustainable Energy}

\end{enumerate}

\subsubsection*{JOURNAL PAPERS: In Press}

\begin{enumerate}
    \item 
A Scalable Optimal Power Management for Large-Scale Battery Energy Storage Systems

\textbf{Amir Farakhor}, Di Wu, Yebin Wang, Huazhen Fang

{\it IEEE Transactions on Transportation Electrification}

\end{enumerate}

\subsubsection*{JOURNAL PAPERS: Published}

\begin{enumerate}
    \item 
A Novel Modular, Reconfigurable Battery Energy Storage System: Design, Control, and Experimentation

\textbf{Amir Farakhor}, Di Wu, Yebin Wang, Huazhen Fang

{\it IEEE Transactions on Transportation Electrification}, 9 (2), pp. 2878–2890, 2023

    \item 	
A Study on an Improved Three-Winding Coupled Inductor Based DC/DC Converter with Continuous Input Current

\textbf{Amir Farakhor}, Mehdi Abapour, Mehran Sabahi, Saeid Gholami Farkoush, Seung-Ryle Oh, Sang-Bong Rhee

{\it Energies}, 13 (7), 2020

    \item 	
Design, Analysis, and Implementation of a Multiport DC–DC Converter for Renewable Energy Applications

\textbf{Amir Farakhor}, Mehdi Abapour, Mehran Sabahi

{\it IET Power Electronics}, 12 (3), pp. 465–475, 2019

    \item 	
Study on the Derivation of the Continuous Input Current High-Voltage Gain DC/DC Converters

\textbf{Amir Farakhor}, Mehdi Abapour, Mehran Sabahi

{\it IET Power Electronics}, 11 (10), pp. 1652–1660, 2018

    \item 	
Design Optimization of a Ćuk DC/DC Converter Based on Reliability Constraints

Amirreza Zarrin Gharehkoushan, Mehdi Abapour, \textbf{Amir Farakhor}

{\it Turkish Journal of Electrical Engineering and Computer Sciences}, 25 (3), pp. 1932–1945, 2017

    \item 	
Symmetric and Asymmetric Transformer Based Cascaded Multilevel Inverter with Minimum Number of Components

\textbf{Amir Farakhor}, Rouzbeh Reza Ahrabi, Hossein Ardi, Sajad Najafi Ravadanegh

{\it IET Power Electronics}, 8 (6), pp. 1052–1060, 2015

    \item 	
A Novel High Step-up DC/DC Converter Based on Integrating Coupled Inductor and Switched-Capacitor Techniques for Renewable Energy Applications

Ali Ajami, Hossein Ardi, \textbf{Amir Farakhor}

{\it IEEE Transactions on Power Electronics}, 30 (8), pp. 4255–4263, 2015

    \item 	
Design, Analysis and Implementation of a Buck–Boost DC/DC Converter

Ali Ajami, Hossein Ardi, \textbf{Amir Farakhor}

{\it IET Power Electronics}, 7 (12), pp. 2902–2913, 2014

    \item 	
Minimisations of Total Harmonic Distortion in Cascaded Transformers Multilevel Inverter by Modifying Turn ratios of the Transformers and Input Voltage Regulation

Ali Ajami, \textbf{Amir Farakhor}, Hossein Ardi

{\it IET Power Electronics}, 7 (11), pp. 2687–2694, 2014

    \item 	
Non-Isolated Multi-Input–Single-Output DC/DC Converter for Photovoltaic Power Generation Systems

Mohammad Reza Banaei, Hossein Ardi, Rana Alizadeh, \textbf{Amir Farakhor}

{\it IET Power Electronics}, 7 (11), pp. 2806–2816, 2014

    \item 	
Analysis and Implementation of a New Single-Switch Buck–Boost DC/DC Converter

Mohammad Reza Banaei, Hossein Ardi, \textbf{Amir Farakhor}

{\it IET Power Electronics}, 7 (7), pp. 1906–1914, 2014

\end{enumerate}

% This resets the counter for the enumerate environment.
% \myEnumReset 

\subsubsection*{CONFERENCE PROCEEDINGS: Under Review}

\begin{enumerate}

    \item 
Optimal Power Management of Battery Energy Storage Systems via Ensemble Kalman Inversion

\textbf{Amir Farakhor}, Iman Askari, Di Wu, Huazhen Fang

{\it American Control Conference (ACC)}, 2024

\end{enumerate}

\subsubsection*{CONFERENCE PROCEEDINGS: Published}

\begin{enumerate}

    \item 
Distributed Optimal Power Management for Battery Energy Storage Systems: A Novel Accelerated Tracking ADMM Approach

\textbf{Amir Farakhor}, Yebin Wang, Di Wu, Huazhen Fang

{\it American Control Conference (ACC)}, 2023

    \item 
A Novel Modular, Reconfigurable Battery Energy Storage System Design

\textbf{Amir Farakhor}, Huazhen Fang

{\it 47th Annual Conference of the IEEE Industrial Electronics Society (IECON)}, 2022

    \item 
Dynamic Modeling and Online Parameter Identification of a Coupled-Inductor-Based DC-DC Converter with Leakage Inductance Effect Consideration

\textbf{Amir Farakhor}, Huazhen Fang

{\it 47th Annual Conference of the IEEE Industrial Electronics Society (IECON)}, 2022

    \item 
A New Coupled Inductor-Based High Step-Up DC-DC Converter for PV Applications

Alireza Eyvazizadeh Khosroshahi, Amin Shotorbani, Hoda Dadashzadeh, \textbf{Amir Farakhor}, Liwei Wang

{\it 20th Workshop on Control and Modeling for Power Electronics (COMPEL)}, 2019

    \item 
A Two-Stage Coupled-Inductor-Based Cascaded DC-DC Converter with a High Voltage Gain

Alireza E. Khosroshahi, Liwei Wang, Hoda Dadashzadeh, Hossein Ardi, \textbf{Amir Farakhor}, Amin Shotorbani

{\it IEEE Canadian Conference of Electrical and Computer Engineering (CCECE)}, 2019

    \item 
Analysis and Design Procedure of a Novel High Voltage Gain DC/DC Boost Converter

\textbf{Amir Farakhor}, Hossein Ardi, Mehdi Abapour

{\it 8th Power Electronics, Drive Systems \& Technologies Conference (PEDSTC)}, 2017

    \item 
Application of Finite Control Set Model based Predictive method for power flow control using Unified Power Flow Controller

\textbf{Amir Farakhor}, Alireza E Khosroshahi, Mehdi Abapour, Saeed Azimi Saadat

{\it 9th International Conference on Electrical and Electronics Engineering (ELECO)}, 2015

    \item 
New Cascaded Multilevel Inverter Topology with Reduced Number of switches and Sources

Seyed Hossein Hosseini, \textbf{Amir Farakhor}, Saeideh Khadem Haghighian

{\it 8th International Conference on Electrical and Electronics Engineering (ELECO)}, 2013

    \item 
Novel Algorithm of Maximum Power Point Tracking (MPPT) for Variable Speed PMSG Wind Generation Systems through Model Predictive Control

Seyed Hossein Hosseini, \textbf{Amir Farakhor}, Saeideh Khadem Haghighian

{\it 8th International Conference on Electrical and Electronics Engineering (ELECO)}, 2013

    \item 
Novel Algorithm of MPPT for PV Array Based on Variable Step Newton-Raphson Method through Model Predictive Control

Seyed Hossein Hosseini, \textbf{Amir Farakhor}, Saeideh Khadem Haghighian

{\it 13th International Conference on Control, Automation and Systems (ICCAS)}, 2013

\end{enumerate}

\myEnumReset 

\subsection*{PATENTS \& APPLICATIONS}
\hrule
\vspace{0.2cm}

\begin{enumerate}

    \item 
A Modular, Reconfigurable Battery Energy Storage System (RBESS)

\textbf{Amir Farakhor}, Huazhen Fang

PCT/US2022/077918, 2022

\end{enumerate}

\myEnumReset 

\subsection*{RESEARCH INTERESTS}
\hrule
\vspace{0.2cm}

%\begin{itemize}
%	\item {\bf Energy Storage Systems}
%	\begin{itemize}
%		\item Design, Optimal Control, and Experimental Demonstration
%	\end{itemize}
%
%	\item {\bf Energy Management of Large-Scale systems}
%	\begin{itemize}
%		\item Grid-Interactive Buildings
%		\item Community-Level Energy Systems
%		\item Outer Space Energy Generation, Distribution, and Management
%	\end{itemize}
%
%	\item {\bf Renewable Energy}
%	\begin{itemize}
%		\item Sustainable Energy Generation and Distribution
%		\item Maximum Power Extraction from Wind and Solar Energy Sources
%		\item Power Electronics
%	\end{itemize}
%
%	\item {\bf Electric Vehicles}
%	\begin{itemize}
%		\item Battery Pack Design and Optimization
%		\item Powertrain Design and Control
%		\item Power Electronics for Electric Vehicle Applications
%	\end{itemize}
%
%	\item {\bf Optimal Control}
%	\begin{itemize}
%		\item Model Predictive Control
%		\item Optimal Control Through Cloud Computing
%		\item Distributed Optimal Control in the Presence of Communication
%	\end{itemize}
%\end{itemize}

\begin{itemize}
	\setlength\itemsep{1em}
	\item{\bf Energy Storage Systems}: Design, optimal control strategies, and experimental validation.	
	\item{\bf Renewable Energy}: Sustainable generation and distribution, efficient power extraction from wind and solar sources, and advancements in power electronics.
	\item{\bf Energy Management}: Large-scale systems, including grid-interactive buildings, community-level energy systems, and energy solutions for outer space.
	\item{\bf Electric Vehicles}: Battery pack, powertrain, and power electronics design and control.
	\item{\bf Power Electronics}: Design, control, and experimentation of various power electronic converters with applications in charging stations, renewable energy generation systems, and advanced grid integration technologies.
	\item{\bf Optimal Control}: Model predictive control, cloud-based optimal control, and distributed control in communication-rich environments, enhancing system efficiency and performance.
\end{itemize}

%\begin{tabbing}
%xxxx\=xxxxxxxx\=xxxxxxxx\=xxxxxxxx\=\kill
%%\>\begin{tabular*}{6.1in}{lr}
%
%\>\begin{tabular*}{0.9\linewidth}{l@{\extracolsep{\fill}}r}
%\textbf{Graduate Research Assistant} & University of Kansas \\
%Research Supervisor: Dr. Huazhen Fang  & Dept. of Mechanical Engineering\\
% & {\it Spring 2022 - Fall 2023}\\ 
% & \\
%
%%\textbf{Graduate Teaching Assistant} & University of Kansas \\
%%\textbf{Course}: Mechanical Engineering Measurements and Experiments & Dept. of Mechanical Engineering\\
%%Teaching Supervisor: Dr. Carl Luchies & {\it Fall 2023}\\ 
%% & \\
%
%\end{tabular*}
%\end{tabbing}

%\vspace{0.1cm}
%\newpage
\subsection*{TEACHING EXPERIENCE}
\hrule
\vspace{0.2cm}

\begin{tabbing}
xxxx\=xxxxxxxx\=xxxxxxxx\=xxxxxxxx\=\kill
%\>\begin{tabular*}{6.1in}{lr}

\>\begin{tabular*}{0.9\linewidth}{l@{\extracolsep{\fill}}r}
\textbf{Lecturer} & University of Kansas \\
\textbf{Course}: EECS 444 Control Systems  & Dept. of Electrical Eng. and Computer Science\\
 & {\it Spring 2024}\\ 
 & \vspace{-5pt}\\

\textbf{Graduate Teaching Assistant} & University of Kansas \\
\textbf{Course}: Mechanical Engineering Measurements and Experiments & Dept. of Mechanical Engineering\\
Teaching Supervisor: Dr. Carl Luchies & {\it Fall 2023}\\ 
 & \vspace{-5pt}\\

\textbf{Volunteer Instructor} & University of Kansas \\
KU Engineering Summer Camp: Control and Robotics  & Dept. of Mechanical Engineering\\
& {\it Summer 2022-2023}\\ 
 & \vspace{-5pt}\\

\textbf{Graduate Teaching Assistant} & University of Kansas \\
\textbf{Course}: Mechanical Engineering Measurements and Experiments & Dept. of Mechanical Engineering\\
Teaching Supervisor: Dr. Carl Luchies & {\it Fall 2021}\\ 
 & \vspace{-5pt}\\

\textbf{Graduate Teaching Assistant} & University of Kansas \\
\textbf{Course}: Mechanical Engineering Measurements and Experiments & Dept. of Mechanical Engineering\\
Teaching Supervisor: Dr. Geng Ku& {\it Spring 2021}\\ 
\end{tabular*}
\end{tabbing}

\subsection*{PRESENTATIONS}
\hrule
\vspace{0.2cm}

\subsubsection*{CONFERENCE TALKS}
\begin{enumerate}
    \item 
       ``Distributed Optimal Power Management for Battery Energy Storage Systems: A Novel Accelerated Tracking ADMM Approach",
       {\it American control Conference (ACC)}, San Diego, California, U.S., May 2023. 
    \item 
       ``Reconfigurable Design of Battery Energy Storage Systems: From Architecture to Control",
       {\it 4th International Conference on Smart Power \& Internet Energy Systems}, Beijing, China, December 2022. 
    \item 
       ``A Novel Modular, Reconfigurable Battery Energy Storage System Design",
       {\it 47th Annual Industrial Electronics Conference (IECON)}, Virtual Conference, October 2021.
    \item 
       ``Dynamic Modeling and Online Parameter Identification of a Coupled-Inductor-Based DC-DC Converter with Leakage Inductance Effect Consideration",
       {\it 47th Annual Industrial Electronics Conference (IECON)}, Virtual Conference, October 2021.
    \item 
       ``Analysis and Design Procedure of a Novel High Voltage Gain DC/DC Boost Converter",
       {\it 8th Power Electronics, Drive Systems \& Technologies Conference (PEDSTC)}, Mashhad, Iran, February 2017.
    \item 
       ``Optimal Integration of Wind Power Resources in Distribution Networks Considering Demand Response Programs",
       {\it 9th International Conference on Electrical and Electronics Engineering (ELECO)}, Bursa, Turkey, November 2015. 
    \item 
       ``Impact of Active Network Management in Operation of Tabriz Distribution System",
       {\it 9th International Conference on Electrical and Electronics Engineering (ELECO)}, Bursa, Turkey, November 2015. 
\end{enumerate}

%\vspace{0.1cm}
\subsection*{HONORS AND AWARDS}
\hrule
\vspace{0.2cm}

{\renewcommand{\arraystretch}{1.25}
\begin{tabular}{llr}
 2023 & \textbf{Recipient}, ACC Student Travel Grant & \textit{American Control Conference} \\
 2023 & \textbf{Recipient}, Tradition of Excellence Award & \textit{University of Kansas} \\
 2023 & \textbf{First Place}, Graduate Engineering Association - Research Showcase & \textit{University of Kansas} \\
 2023 & \textbf{First Place}, Research Symposium of the Inst. for Information Sciences (I2S) & \textit{University of Kansas} \\
 2023 & \textbf{Presenter}, Capital Graduate Research Summit (CGRS) & \textit{University of Kansas} \\
 2022 & \textbf{Student of the Year}, Information and Smart Systems Laboratory (ISSL) & \textit{University of Kansas} \\
 2022 & \textbf{Winner}, KU Engineering Research Showcase (Poster Presentation) & \textit{University of Kansas} \\
 2022 & \textbf{Third Place}, KU Engineering Research Showcase (Virtual Presentation) & \textit{University of Kansas} \\
\end{tabular}
}


%\begin{itemize}
%	\item Technical Program Committee Member, QoS-IP, Milano, Italy, 2003.
%	\item Member Network Processing Forum (Originally CPIX) - 2000-2001.
%\end{itemize}

%\vspace{0.1cm}
\subsection*{ACADEMIC SERVICE}
\hrule
\vspace{0.2cm}

{\renewcommand{\arraystretch}{1.25}
\begin{tabular}{rl}
 \textbf{Publicity Chair} & 7th IEEE International Conference on Industrial Cyber-Physical Systems (ICPS) - 2024 \\
 \textbf{Reviewer} & IEEE Transactions of Power Electronics (20 Reviews) \\
 \textbf{Reviewer} & IEEE Transactions of Industrial Electronics (32 Reviews) \\
 \textbf{Reviewer} & IEEE Transactions of Energy Conversion (17 Reviews) \\
 \textbf{Reviewer} & IEEE Open Journal of Industrial Electronics Society (11 Reviews) \\
 \textbf{Reviewer} & IEEE Transactions of Vehicular Technology (2 Reviews) \\
 \textbf{Reviewer} & IEEE Transactions of Transportation Electification (2 Reviews) \\
 \textbf{Reviewer} & IEEE Transactions of Industrial Applications (1 Reviews) \\
 \textbf{Reviewer} & International Transactions in Electrical Energy Systems (3 Reviews) \\
 \textbf{Reviewer} & IEEE Control Systems Letters (3 Reviews) \\
\end{tabular}
}

%\begin{itemize}
%	\item \textbf{Publicity Chair} \hspace{1cm} The 7th IEEE International Conference on Industrial Cyber-Physical Systems (ICPS) - 2024
%	\item \textbf{Reviewer} \hspace{1cm} IEEE Transactions of Power Electronics (20 Reviews)
%	\item \textbf{Reviewer} \hspace{1cm} IEEE Transactions of Industrial Electronics (26 Reviews)
%	\item \textbf{Reviewer} \hspace{1cm} IEEE Transactions of Energy Conversion (16 Reviews)
%	\item \textbf{Reviewer} \hspace{1cm} IEEE Open Journal of Industrial Electronics Society (8 Reviews)
%	\item \textbf{Reviewer} \hspace{1cm} IEEE Transactions of Vehicular Technology (2 Reviews)
%	\item \textbf{Reviewer} \hspace{1cm} IEEE Transactions of Transportation Electification (2 Reviews)
%\end{itemize}

% \begin{multicols}{3}[\section*{Long title across all the columns}]

%\vspace{0.1cm}
%\subsection*{LANGUAGES}
%\hrule
%\vspace{0.2cm}
%\begin{list}{}{}
%	\item  Proficient in English, Hindi and Marathi. Working knowledge of Tamil and German.
%\end{list}

%\newpage

\vspace{0.1cm}
%\begin{multicols}{2} [\subsection*{REFERENCES}]

%\newpage

\subsection*{REFERENCES}
\hrule

%\vspace{0.2cm}
%\subsubsection*{FROM ACADEMIA}

\begin{footnotesize}

\begin{multicols}{2} 
\noindent 
Dr. Huazhen Fang \\
Associate Professor \\
Dept. of Mechnical Engineering \\
University of Kansas, Lawrence, KS \\
fang@ku.edu \\

\noindent
Dr. Di Wu\\
Chief Engineer and Team Leader \\
Optimization and Control Group \\
Pacific Northwest National Laboratory (PNNL), Richland, WA \\
di.wu@pnnl.gov \\

\columnbreak

\noindent 
Dr. Carl Luchies \\
Associate Professor \\
Dept. of Mechnical Engineering \\
University of Kansas, Lawrence, KS \\
luchies@ku.edu \\

\noindent
Dr. Geng Ku \\
Laboratory Manager/Staff Scientist \\
Dept. of Mechnical Engineering \\
California Institute of Technology (Caltech) \\
gku@caltech.edu \\

\end{multicols}

%\subsubsection*{FROM INDUSTRY}
%\begin{multicols}{2} 
%
%\noindent
%Dr. Flavio Bonomi\\
%Chief Architect \\
%Cisco Systems \\
%170 West Tasman Dr. \\
%San Jose, CA 95134 \\
%Phone: (408) 526-4085 \\
%flavio@cisco.com \\
%
%\columnbreak
%\noindent
%Mr. Ajit Shelat \\
%Chief Technology Officer \\
%SwitchOn Networks (now PMC-Sierra) \\
%5 Mantri Mansion, Vrindavan Society \\
%Pashan, Pune, 411 008, India \\
%Phone: +91-20-589-8412 \\
%ashelat@vsnl.com\\
%
%\end{multicols}
%\begin{tabbing}
%xxxx\=xxxxxxxxxxxxxxxxxxxxx\=xxxx\=xxxxxxxxxxxxxxxxxxxxxxxxxxxxxxxxxxxxxxxxx\=\kill
%\>{Nick McKeown}\> :\>nickm@stanford.edu\\
%\>{Balaji Prabhakar}\>:\>balaji@stanford.edu\\
%\>{Ajit Shelat}\>:\>ajit\_shelat@pmc-sierra.com\\
%\>{S.S.S.P. Rao}\>:\>ssspr@cse.iitb.ernet.in\\
%\end{tabbing}
%\newpage

\end{footnotesize}
\end{small}
\end{document}
%%%%%%%%%%%%%%%%%%%%%%%%%%%%%%%%%%%%%%%%%%%%%%%%%%%%%%%%%%%%%%%%%%%%%%%%